Current and next-generation precision neutrino oscillation experiments aim to determine the neutrino mass ordering, to measure the extent of CP violation in the lepton sector and to probe beyond Standard Model physics, such as sterile neutrinos. These experiments rely on models for neutrino interactions with matter that are complicated by nuclear effects and final state interactions. To date, these effects cannot be modelled precisely, especially for heavy target nuclei, typically used in modern neutrino experiments. Many future experiments will employ detectors based on the liquid argon time projection chamber technology. As a consequence, neutrino-argon cross-section measurements are of the outmost importance, especially given the relative scarcity of neutrino-argon data, due to the novelty of the technology.

This thesis presents the first measurement of muon-neutrino charged-current inclusive cross section on argon at a mean neutrino energy of 0.8 GeV.
Data were collected using the MicroBooNE liquid argon time projection chamber in the Fermilab Booster neutrino beam during a period of six months, corresponding to an exposure of $1.6 \times 10^{20}$ protons on target. The measured cross section is presented as a function of the outgoing muon momentum, using multiple Coulomb scattering as a measurement technique, as well as the muon angle with respect to the beam direction. A comparison of the measured cross section obtained with two different configurations of the \textsc{Genie} neutrino event generator is performed, and better agreement is found when using nuclear-effect modelling in the theoretical calculations. Additionally, the total flux integrated cross section is measured to be $0.693 \pm 0.010 \, (\text{stat.}) \pm 0.165 \, (\text{syst.}) \times 10^{-38} \, \text{cm}^{2}$.


