\chapter{Introduction}
\label{ch:introduction}





Particle physics studies the building blocks of our Universe and the elementary forces that bind them together and make them interact to form all the matter that we know of: from the particles on Earth to the far-away stars. Remarkably, all the elementary particles and forces are consistently described by a single theory: the \acrfull{sm} of particle physics. The \acrshort{sm} is able to explain all the particles that have currently been discovered, the way they obtain their mass thanks to the Higgs field, and the way they interact with one another. For all of this, it is definitely one of the most significant achievements of humankind. 

At the same time, from cosmological and astrophysical studies it is known that matter only constitutes 5\% of the known universe. The remaining 95\% is made by the so-called dark matter (25\%) and dark energy (70\%). The \acrshort{sm}, in its current formulation, does not predict other particles that could explain dark matter.
Many other questions exist that make physicists wonder about the correctness, or at the least on the completeness, of the \acrshort{sm}.
Fixing, or completing the \acrshort{sm} is extremely challenging, as it makes remarkable predictions of all the experimentally observed phenomena. For example, the \acrshort{sm} has been confirmed at the level of one part per trillion precision by looking at the comparison between the measured and predicted magnetic moment of the electron \cite{electron_magnetic_moment}. This is the most accurately verified prediction in the history of physics. 

The \acrshort{sm} predicts the existence of antimatter, that arises as a direct consequence of combining two of the most fundamental known concepts in physics, the theory of relativity and quantum mechanics. On the other hand, the atoms in our local region of the Universe are formed from electrons, protons and neutrons rather than their equivalent antiparticles. The possibility that there are galaxies or regions of space dominated by antimatter can be excluded by the astronomical searches of photons from the $e^{+}e^{-}$ annihilation process that would occur at the interfaces between matter and antimatter dominated regions of the Universe~\cite{antimatter}. The predominance of matter is believed to have arisen in the early evolution of the Universe, and this asymmetry between matter and antimatter is among the most pressing open questions in particle physics.

Moreover, several other questions remain unanswered: how to unify the forces? Why are there three generations of particles? Why the observed pattern of particles masses? And more.

Ultimately, the \acrshort{sm} is in fact just a model: while it can predict all the discovered particles and their interactions, it still has 28 parameters that are not fixed but need to be measured by experiments (like the particle masses and the interaction strengths).

Current and future experiments will aim to address these open questions and neutrinos seem very promising to solve many of the great puzzles of physics. Despite being the most abundant particles just after photons, neutrinos remain the least understood ones. Indeed, several neutrino properties are still unknown, like their absolute mass, the number of species, or their exact nature. The \acrshort{sm} predicts neutrinos as massless particles, but experiments \cite{superk, sno} have demonstrated the existence of neutrino oscillations, where neutrinos have been observed changing in one another, implying neutrinos as massive particles. This is already strongly challenging the \acrshort{sm}.

The asymmetry between matter and antimatter could be explained by a symmetry being broken in our Universe. In the \acrshort{sm}, this symmetry is called \acrshort{cp}: if \acrshort{cp} symmetry is violated, that means that physics draws a distinction between matter and antimatter, and could explain what causes our universe to be matter-dominated. While \acrshort{cp} violation has been observed in the quark sector, it is not strong enough to justify the current matter/antimatter asymmetry. Alternatively, \acrshort{cp} symmetry is yet to be observed in the neutrino sector, and this is an auspicious place to look.

An essential missing piece in the understanding of neutrinos is the knowledge of the neutrino mass, but especially the knowledge of the mechanism that gives origin to this mass. 
Mass mechanisms require that neutrinos are either ``Dirac'' or ``Majorana'' in nature. Understanding their nature is therefore of fundamental importance as it has direct implications on the understanding of the neutrino mass.

To unravel all these mysteries around neutrinos, precision neutrino oscillation experiments are needed. In the near future, these experiments will be able to answer most of the above questions and to search for physics beyond the \acrshort{sm}.  
Precision measurements require incredibly high-resolution detectors that are able to resolve all the subatomic particles produced from a neutrino interaction. Liquid Argon Time Projection Chambers (\acrshort{lartpc}s) are cutting-edge detectors now employed for neutrino physics, and their technology allows one to track particles produced by neutrino interactions to unprecedented, millimetre scale, 3D resolution.

Precision neutrino experiments, in turn, require precise knowledge of neutrino interactions with matter. While these have been studied for a long time, modern detectors like \acrshort{lartpc} use complex nuclei as target material, which complicate the study of neutrino interactions. Many of the current neutrino experiments dedicate a considerable effort in the study of neutrino interactions, which will then be a crucial point in the study of neutrino oscillations.
MicroBooNE, a neutrino detector at Fermilab (U.S.A.), employs the \acrshort{lartpc} detector technology for the study of neutrinos and can measure neutrino-argon interactions with unprecedented precision.

This thesis presents the first measurement of muon neutrino charged current interactions on argon at low neutrino energy ($E_\nu \sim 0.8$ GeV). Most of the future neutrino experiments will use argon as target material, and this measurement is of fundamental importance for the successful completion of these experiments. The analysis described in this thesis
uses data collected by the MicroBooNE experiment from February to October 2016.





































