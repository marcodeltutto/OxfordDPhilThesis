\chapter{Conclusions and Outlook}
\label{ch:conclusions}



The current and next generation precision neutrino oscillation experiments aim to determine the neutrino mass ordering, to measure the extent of CP violation in the lepton sector, and to probe beyond Standard Model physics, such as sterile neutrinos.  These experiments rely on models for neutrino interactions with matter complicated by the existence of nuclear effects and final state interactions. Moreover, neutrino oscillation experiments measure the oscillation probability which depends on the neutrino energy, and therefore requires a link between the energy of reconstructed secondary hadrons and leptons emerging from the neutrino's interaction to its initial energy. This link is also affected by the existence of nuclear effects and final state interactions, which to date cannot be modelled exactly, especially for heavy target nuclei typically used in modern neutrino experiments. 
As shown in Chapter~\ref{ch:neutrino_interactions}, neutrino interaction modelling uncertainties are already the dominant systematic on current measurements of the oscillation parameters and will soon become the principal limitation if an improved understanding will not be achieved. In particular, it is essential to better understand the nuclear effects that obfuscate any attempt to measure the cross section of interactions between neutrinos and free nucleons.

Over the past years, many experiments have measured the total inclusive charge-current cross section for neutrino ($\nu_\mu N \rightarrow \mu^- X$) scattering off nucleons covering a broad range of neutrino energies. To provide a more complete picture, differential cross sections for inclusive scattering processes are preferred. Inclusive differential measurements have been performed on iron by NuTeV~\cite{NuTeV}, on lead, iron, and carbon by MINERvA~\cite{MINERvACCincl3}, on carbon by T2K~\cite{t2k_ccinclusive_2018} and on argon by ArgoNeuT~\cite{ArgoNeuTCCincl, ArgoNeuTCCincl2}. ArgoNeuT is the only experiment that published cross sections for neutrino-argon scattering to date, at a neutrino energy around 5 GeV.
Many future experiments will employ liquid argon time projection chambers as detectors. As a consequence, neutrino-argon cross-section measurements are of paramount importance, especially given the relative scarcity of neutrino-argon data.

This thesis presented the first measurement of differential muon-neutrino charged-current inclusive cross sections on argon at a mean neutrino energy of 0.8 GeV.
Data were collected using the MicroBooNE liquid argon time projection chamber in the Fermilab Booster neutrino beam for a period of six months and corresponding to $1.6 \times 10^{20}$ protons on target of exposure. The measured cross section is presented as a function of muon momentum and muon angle with respect to the beam direction, making it a double-differential cross section. This analysis has full angular coverage, achieved thanks to the $4\pi$ acceptance of liquid argon time projection chamber detectors and uses multiple Coulomb scattering for measuring the muon momentum. The results for the double-differential cross section are in Figure~\ref{fig:3d_xsec} and ~\ref{fig:trkcostheta_trkmumom__xsec_anglesplit}.
The measured cross section is compared with two different configurations of the \textsc{Genie} neutrino event generator, and better agreement is found with theory calculations that include a nuclear effect modelling. Additionally, single-differential cross sections in muon momentum and angle have also been presented separately, and a total flux integrated cross section $\sigma(\nu_\mu + \text{Ar} \rightarrow \mu^- + X)$ per nucleon of $0.693 \pm 0.010 \, (\text{stat.}) \pm 0.165 \, (\text{syst.}) \times 10^{-38} \, \text{cm}^{2}$ is obtained.
An estimation of the discriminating power between theoretical models was also provided, through a $\chi^2$ test between data and two different predictions.

Improvements to the analysis are planned for future iterations, together with a more precise understanding of the detector systematic uncertainties, which will improve the precision of this measurement. New data samples are currently being prepared with the aim to reduce both statistical and systematic uncertainties, by using a data-driven model of the detector and by exploiting an external cosmic-ray tagger, recently installed~\cite{crt}. 
Further structural changes in the analysis could also lead to improvements. The smearing matrices are binned in momentum and angle, but more information could be included by binning variables that affect reconstruction, for example the angular separation of tracks or total particle multiplicity. Moreover, model parameters can be constrained in a fit to the measured cross section.

Besides the physics itself, this inclusive measurement has a huge value for the development of reconstruction and selection tools in MicroBooNE. Since the detector is located on the surface, it is constantly pierced by cosmic rays, which constitute the main background when looking for neutrino-induced muon tracks. In Chapter~\ref{ch:reconstruction}, many techniques were described to distinguish neutrino-induced particles from cosmic-ray muons for a sample with full angular coverage. 
The matching between optical and track information was also described. This is the first time that such flash matching is performed in an automated way in a liquid argon time projection chamber detector demonstrating the capability of such detector technology.
Moreover, the final neutrino enriched sample used for this analysis (selected using the event selection described in Section~\ref{ch:event_selection}) is currently being used as a pre-selection for more exclusive channels.




